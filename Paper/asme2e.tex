%%%%%%%%%%%%%%%%%%%%%%%%%%% asme2e.tex %%%%%%%%%%%%%%%%%%%%%%%%%%%%%%%
% Template for producing ASME-format articles using LaTeX            %
% Written by   Harry H. Cheng                                        %
%              Integration Engineering Laboratory                    %
%              Department of Mechanical and Aeronautical Engineering %
%              University of California                              %
%              Davis, CA 95616                                       %
%              Tel: (530) 752-5020 (office)                          %
%                   (530) 752-1028 (lab)                             %
%              Fax: (530) 752-4158                                   %
%              Email: hhcheng@ucdavis.edu                            %
%              WWW:   http://iel.ucdavis.edu/people/cheng.html       %
%              May 7, 1994                                           %
% Modified: February 16, 2001 by Harry H. Cheng                      %
% Modified: January  01, 2003 by Geoffrey R. Shiflett                %
% Use at your own risk, send complaints to /dev/null                 %
%%%%%%%%%%%%%%%%%%%%%%%%%%%%%%%%%%%%%%%%%%%%%%%%%%%%%%%%%%%%%%%%%%%%%%

%%% use twocolumn and 10pt options with the asme2e format
\documentclass[twocolumn,10pt]{asme2e}
\special{papersize=8.5in,11in}

%% The class has several options
%  onecolumn/twocolumn - format for one or two columns per page
%  10pt/11pt/12pt - use 10, 11, or 12 point font
%  oneside/twoside - format for oneside/twosided printing
%  final/draft - format for final/draft copy
%  cleanfoot - take out copyright info in footer leave page number
%  cleanhead - take out the conference banner on the title page
%  titlepage/notitlepage - put in titlepage or leave out titlepage
%  
%% The default is oneside, onecolumn, 10pt, final

%%% Replace here with information related to your conference
\confshortname{PowerEnergy2017}
\conffullname{the ASME 2017 Power and Energy Conference \&\\
            }

%%%%% for date in a single month, use
%\confdate{24-28}
%\confmonth{September}
%%%%% for date across two months, use
\confdate{June 26-30}
\confyear{2017}
\confcity{Charlotte, North Carolina}
\confcountry{USA}

%%% Replace DETC2009/MESA-12345 with the number supplied to you 
%%% by ASME for your paper.
\papernum{PowerEnergy2017-3673}

%%% You need to remove 'DRAFT: ' in the title for the final submitted version.
\title{DRAFT: Sugarcane Bagasse in Belize: An Economic Assessment and Overview of Potential Opportunities}

%%% first author
\author{Aldair E. Gongora
    \affiliation{
	%Integration Engineering Laboratory\\
	Department of Mechanical Engineering\\
	Boston University\\
	Boston, Massachusetts 02215\\
    Email: agongora@bu.edu
    }	
}

%%% second author
%%% remove the following entry for single author papers
%%% add more entries for additional authors
\author{Dorien O. Villafranco %\thanks{Address all correspondence to this author.} \\
       %{\tensfb Second Coauthor}     
    \affiliation{Department of Mechanical Engineering\\
	Boston University\\
	Boston, Massachusetts, 02215\\
	Email: dvillafr@bu.edu
    }
}

\begin{document}

\maketitle    

%%%%%%%%%%%%%%%%%%%%%%%%%%%%%%%%%%%%%%%%%%%%%%%%%%%%%%%%%%%%%%%%%%%%%%
\begin{abstract}
{\it Sugarcane is the most important crop for the economy of Belize. With sugarcane deliveries exceeding 122 thousand metric tons in the fourth quarter of 2015, sugar is Belize's largest contributor to the agricultural sector with exports approximating BZ\$140,000,000 for the year 2015. With commissioning of the Belize Co-generation Energy (Belcogen) plant completed in December 2009, sugarcane bagasse has been used for electric energy production. Currently, the single co-generation plant in the country contributes about $15\%$ of electricity to the national grid. Belize currently imports about $45\%$ of its electric energy needs from Mexico, and to date, this is the country's most reliable energy source. With energy needs of the country projected to rise at a rate of $4\%$ per annum, and with the costly import of energy, there exists the need to explore the expansion of co-generation energy technologies to increase local energy generation output to the national grid. The aim of this paper is to demonstrate the positive conditions which support such an expansion by analyzing the current co-generation technologies in the context of necessary economic, regulatory and technical pathways toward this increase of output. This is done through a discussion of various economic and technical scenarios with recommendations made for the most feasible option. The introduction of a new sugar plant in Belize with projections of annually producing 100,000 metric tons by 2020 and with intentions of introducing co-generation technologies could lead to $20\%$ of local electrical energy being supplied to the national grid. It is projected that a new sugar plant of this size would grow the agricultural gross domestic product of the country by approximately $4\%$. These benefits further support the need for a discussion of co-generation technologies and the associated positive conditions supporting the expansion of the aforementioned technologies. Additionally, it is also relevant to mention the feasibility of alternative renewable energy technologies that could contribute to Belize's national goal of increasing energy efficiency, sustainability, and resilience over the next 30 years.}
\end{abstract}

%%%%%%%%%%%%%%%%%%%%%%%%%%%%%%%%%%%%%%%%%%%%%%%%%%%%%%%%%%%%%%%%%%%%%%
%\begin{nomenclature}
%\entry{A}{You may include nomenclature here.}
%\entry{$\alpha$}{There are two arguments for each entry of the nomemclature environment, the symbol and the definition.}
%\end{nomenclature}

%The spacing between abstract and the text heading is two line spaces.  The primary text heading is  boldface in all capitals, flushed left with the left margin.  The spacing between the  text and the heading is also two line spaces.

%%%%%%%%%%%%%%%%%%%%%%%%%%%%%%%%%%%%%%%%%%%%%%%%%%%%%%%%%%%%%%%%%%%%%%
\section*{INTRODUCTION}

The increasing demand for energy is a global conundrum with challenges in areas of security, the environment, and economic and social development.\cite{asmemanual} Population growth throughout the world has brought with it the continuous surge in energy demand. Though small, the Central American nation of Belize is no exception to this issue. With a population of about 350,000, growing at a rate of 2.4\%, the country's energy demand is expected to rise at a rate of 4\% per annum \cite{sib_pop}. It has been estimated that within the next 20 years, Belize will need to increase energy generation by about 80\% to satisfy this growing demand \cite{idb_energy}. The country currently receives majority of its electricity supply from Mexico through a direct line into its national grid. This source constitutes 42\% of the demand for electricity, and is to date, the country's most reliable source of energy. For its size, Belize has a relatively high electric energy demand. The country's peak demand is approximately 96 MW \cite{bel_report}. The costly import of electricity from Mexico has prompted the country to explore several renewable energy initiatives. This started with the Chalillo and Mollejon hydro-electric systems. These systems contribute about 37 MW of power to the Belize's national grid which is 39\% of the grid's total demand in peak hours \cite{bel_eng_policy}.

Belize's finances are rooted in the agricultural industry. The sugar industry in Belize is over 150 years old and in the mid 1990's substantially upgraded its capacity to 120, 000 tonnes of sugar per year [7]. Section \# of this paper briefly presents a historical perspective of the Belize sugar industry and its establishment as an imperative proponent in supplying Belize's energy demand. 

The Belize Sugar Industry reported that over 1.1 million tonnes of ground sugarcane were cut in the 2015-2016 crop with approximately 140,000 tons of raw sugar being produced. The plant operated at a factory time efficiency of 97\% with a 95\% Pol extraction for the same crop year [8]. With the production of sugar from ground sugarcane, approximately 3 tons of wet bagasse is produced for every 10 tons of sugarcane crushed[9]. This would amount to approximately 330,000 tons of wet bagasse produced by the plant for the 2015-2016 crop year. As a result of the quantity of bagasse produced and support of renewable energy sources, 35,500,000 US dollars were invested in the construction of the country's first bagasse fired co-generation plant (Belcogen) [10]. The plant was successfully commissioned in 2009 and has a design output of 32.5 MW with approximately 13 MW of electricity supplied to the national grid, which is approximately 14\% of the national grid's needs \cite{asr_report}. Section \# of this paper outlines the economic, regulatory and technical factors affecting the sugar cane plant owned by the Belize Sugar Industry.  Recommendations for improvements in these areas  are presented and their potential effect on the Belizean economy are outlined.

With efforts in further expanding the agricultural sector, the Santander Group in 2008 began an investment of 150 million US dollars in the construction and operation of a new sugar mill in Belize. The sugar mill became fully operational on dd/mm/yy and it is projected to contribute x\% to Belize's economy. Moreover, they intend to further invest in a co-generation power plant which could lead to a supply of 20\% of local electricity to the national grid [12]. Section 4 of this paper outlines, in a similar fashion, the economic, regulatory and technical pathways toward growth of the Santander plant, and the corresponding effects on the Belizean economy.  

Recent economic strains have prompted the rise in electricity rates in the country. In that regard, the discussion to improve and expand on the current renewable energy sources, especially as it relates to energy produced from sugarcane bagasse, has been at the forefront of the country's outlook. The paper attempts to contribute to the field by exploring  recent technical developments  in the renewable energy sector of Belize and forming correlations to the economy  of the country, thus, presenting an argument for continued development of renewable  energy technologies in the country. 


%This article illustrates preparation of ASME paper using \LaTeX2\raisebox{-.3ex}{$\epsilon$}. The \LaTeX\  macro \verb+asme2e.cls+, the {\sc Bib}\TeX\ style file \verb+asmems4.bst+, and the template \verb+asme2e.tex+ that create this article are available on the WWW  at the URL address \verb+http://iel.ucdavis.edu/code/+. To ensure compliance with the 2003 ASME MS4 style guidelines  \cite{asmemanual}, you should modify neither the \LaTeX\ macro \verb+asme2e.cls+ nor the {\sc Bib}\TeX\ style file \verb+asmems4.bst+. By comparing the output generated by typesetting this file and the \LaTeX2\raisebox{-.3ex}{$\epsilon$} source file, you should find everything you need to help you through the preparation of ASME paper using \LaTeX2\raisebox{-.3ex}{$\epsilon$}. Details on using \LaTeX\ can be found in \cite{latex}. Instructions for submitting an electronic version of a paper via ftp for publication on CD-ROM or online  are given at the URL address \verb+http://www.asme.org/pubs/submittal.html+.
%%%%%%%%%%%%%%%%%%%%%%%%%%%%%%%%%%%%%%%%%%%%%%%%%%%%%%%%%%%%%%%%%%%%%
\section*{BELIZE'S SUGAR INDUSTRY}
Sugarcane is a crop rooted deep in Belize's history. The crop was introduced to the country by Mexican immigrants fleeing the Caste War in 1847. For a period of time, these immigrants, mostly in the north of the country, continued to grow sugar for local consumption. Initially interested in logwood export from Belize, British colonists at the time began to take notice of the increasingly popular crop, and took over sugarcane growth and sugar production in the country. By 1857, the first 100 barrels of sugar produced in Belize was exported to Liverpool, England. The industry continued growing marginally until the late 1990s. Technological advances in the milling, and production process bolstered the sugar production over the years. Today, optimal output results in approximately 123,000 tons of sugar being produced each year by the main factory in the country [13]. 


Sugar plays a large role in the economy of the country. In 2012, the economy was projected to grow by a very optimistic 5.2\% due to technological improvements in agricultural practices and very favorable sugar prices within the EU market. Unfavorable weather conditions that year severely affected the agricultural output, and despite a growth in the tourism industry that year the actual economic growth was recorded at only 0.7\%. The economy of Belize has been in decline in recent years. For the past two quarters, the gross domestic product of the country has fallen by approximately 1.6\%. Due to the limited diversity in export products, the economy is heavily reliant on its agricultural output. The growth of the sugar industry, the country's main export crop, is therefore vital to economic growth [14]. 


Despite experiencing such a large growth in recent years, Belize's sugar industry faces several difficulties many of which are related to the economic, regulatory and technical considerations highlighted in this paper. One striking concern lies in the fact that Belize's sugar cane productivity is among the lowest in the world. Compared to other countries in Central America, Belize's sugar cane yield is about 50\% less [15]. 


Recent investments by the Santander group has resulted in the a second sugar production site being recently formed in the country. The group's sugar mill began operations in February 2016, and the first export of 6,250 tonnes of raw sugar was made in July 2016. It is expected that continued growth in the country's production have a direct impact in the growth of the Belizean economy. Santander also has intentions of investing in a co-generation plant which will increase the production of electric energy from sugarcane bagasse from 15\% to 20\% thus reducing the need to import large amounts of costly electric energy from Mexico.  



%%%%%%%%%%%%%%%%%%%%%%%%%%%%%%%%%%%%%%%%%%%%%%%%%%%%%%%%%%%%%%%%%%%%%%
\section*{VERY VERY VERY VERY VERY VERY VERY VERY LONG HEADING}

If the heading should run into more than one line, the run-over is flush left.

%%%%%%%%%%%%%%%%%%%%%%%%%%%%%%%%%%%%%%%%%%%%%%%%%%%%%%%%%%%%%%%%%%%%%%
\subsection*{Second-Level Heading}

The next level of heading is boldface with upper and lower case letters. The heading is flushed left with the left margin. The spacing to the next heading is two line spaces.

%%%%%%%%%%%%%%%%%%%%%%%%%%%%%%%%%%%%%%%%%%%%%%%%%%%%%%%%%%%%%%%%%%%%%%
\subsubsection*{Third-Level Heading.}

The third-level of heading follows the style of the second-level heading, but it is indented and followed by a period, a space, and the start of corresponding text.

%%%%%%%%%%%%%%%%%%%%%%%%%%%%%%%%%%%%%%%%%%%%%%%%%%%%%%%%%%%%%%%%%%%%%%
\section*{PAPER NUMBER}

ASME assigns each accepted paper with a unique number. Replace {\bf DETC98/DAC-1234} in the input file preamble (the location will be obvious) with the paper number supplied to you  by ASME for your paper.


%%%%%%%%%%%%%%%%%%%%%%%%%%%%%%%%%%%%%%%%%%%%%%%%%%%%%%%%%%%%%%%%%%%%%%
\section*{USE OF SI UNITS}

An ASME paper should use SI units.  When preference is given to SI units, the U.S. customary units may be given in parentheses or omitted. When U.S. customary units are given preference, the SI equivalent {\em shall} be provided in parentheses or in a supplementary table. 
%%%%%%%%%%%%%%%%%%%%%%%%%%%%%%%%%%%%%%%%%%%%%%%%%%%%%%%%%%%%%%%%%%%%%%
\section*{MATHEMATICS}

Equations should be numbered consecutively beginning with (1) to the end of the paper, including any appendices.  The number should be enclosed in parentheses and set flush right in the column on the same line as the equation.  An extra line of space should be left above and below a displayed equation or formula. \LaTeX\ can automatically keep track of equation numbers in the paper and format almost any equation imaginable. An example is shown in Eqn.~(\ref{eq_ASME}). The number of a referenced equation in the text should be preceded by Eqn.\ unless the reference starts a sentence in which case Eqn.\ should be expanded to Equation.

\begin{equation}
f(t) = \int_{0_+}^t F(t) dt + \frac{d g(t)}{d t}
\label{eq_ASME}
\end{equation}

%%%%%%%%%%%%%%%%%%%%%%%%%%%%%%%%%%%%%%%%%%%%%%%%%%%%%%%%%%%%%%%%%%%%%%
\section*{FIGURES AND TABLES}

All figures should be positioned at the top of the page where possible.  All figures should be numbered consecutively and captioned; the caption uses all capital letters, and centered under the figure as shown in Fig.~\ref{figure_ASME}. All text within the figure should be no smaller than 7~pt. There should be a minimum two line spaces between figures and text. The number of a referenced figure or table in the text should be preceded by Fig.\ or Tab.\ respectively unless the reference starts a sentence in which case Fig.\ or Tab.\ should be expanded to Figure or Table.


%%%%%%%%%%%%%%%%%%%%%%%%%%%%%%%%%%%%%%%%%%%%%%%%%%%%%%%%%%%%%%%%%%%%%%
%%%%%%%%%%%%%%%% begin figure %%%%%%%%%%%%%%%%%%%
\begin{figure}[t]
\begin{center}
\setlength{\unitlength}{0.012500in}%
\begin{picture}(115,35)(255,545)
\thicklines
\put(255,545){\framebox(115,35){}}
\put(275,560){Beautiful Figure}
\end{picture}
\end{center}
\caption{THE FIGURE CAPTION USES CAPITAL LETTERS.}
\label{figure_ASME} 
\end{figure}
%%%%%%%%%%%%%%%% end figure %%%%%%%%%%%%%%%%%%% 
%%%%%%%%%%%%%%%%%%%%%%%%%%%%%%%%%%%%%%%%%%%%%%%%%%%%%%%%%%%%%%%%%%%%%%


%%%%%%%%%%%%%%%%%%%%%%%%%%%%%%%%%%%%%%%%%%%%%%%%%%%%%%%%%%%%%%%%%%%%%%
%%%%%%%%%%%%%%% begin table   %%%%%%%%%%%%%%%%%%%%%%%%%%
\begin{table}[t]
\caption{THE TABLE CAPTION USES CAPITAL LETTERS, TOO.}
\begin{center}
\label{table_ASME}
\begin{tabular}{c l l}
& & \\ % put some space after the caption
\hline
Example & Time & Cost \\
\hline
1 & 12.5 & \$1,000 \\
2 & 24 & \$2,000 \\
\hline
\end{tabular}
\end{center}
\end{table}
%%%%%%%%%%%%%%%% end table %%%%%%%%%%%%%%%%%%% 
%%%%%%%%%%%%%%%%%%%%%%%%%%%%%%%%%%%%%%%%%%%%%%%%%%%%%%%%%%%%%%%%%%%%%%

All tables should be numbered consecutively and  captioned; the caption should use all capital letters, and centered above the table as shown in Table~\ref{table_ASME}. The body of the table should be no smaller than 7 pt.  There should be a minimum two line spaces between tables and text.

%%%%%%%%%%%%%%%%%%%%%%%%%%%%%%%%%%%%%%%%%%%%%%%%%%%%%%%%%%%%%%%%%%%%%%
\section*{FOOTNOTES\protect\footnotemark}
\footnotetext{Examine the input file, asme2e.tex, to see how a footnote is given in a head.}

Footnotes are referenced with superscript numerals and are numbered consecutively from 1 to the end of the paper\footnote{Avoid footnotes if at all possible.}. Footnotes should appear at the bottom of the column in which they are referenced.


%%%%%%%%%%%%%%%%%%%%%%%%%%%%%%%%%%%%%%%%%%%%%%%%%%%%%%%%%%%%%%%%%%%%%%
\section*{CITING REFERENCES}

%%%%%%%%%%%%%%%%%%%%%%%%%%%%%%%%%%%%%%%%%%%%%%%%%%%%%%%%%%%%%%%%%%%%%%
The ASME reference format is defined in the authors kit provided by the ASME.  The format is:

\begin{quotation}
{\em Text Citation}. Within the text, references should be cited in  numerical order according to their order of appearance.  The numbered reference citation should be enclosed in brackets.
\end{quotation}

The references must appear in the paper in the order that they were cited.  In addition, multiple citations (3 or more in the same brackets) must appear as a `` [1-3]''.  A complete definition of the ASME reference format can be found in the  ASME manual \cite{asmemanual}.

The bibliography style required by the ASME is unsorted with entries appearing in the order in which the citations appear. If that were the only specification, the standard {\sc Bib}\TeX\ unsrt bibliography style could be used. Unfortunately, the bibliography style required by the ASME has additional requirements (last name followed by first name, periodical volume in boldface, periodical number inside parentheses, etc.) that are not part of the unsrt style. Therefore, to get ASME bibliography formatting, you must use the \verb+asmems4.bst+ bibliography style file with {\sc Bib}\TeX. This file is not part of the standard BibTeX distribution so you'll need to place the file someplace where LaTeX can find it (one possibility is in the same location as the file being typeset).

With \LaTeX/{\sc Bib}\TeX, \LaTeX\ uses the citation format set by the class file and writes the citation information into the .aux file associated with the \LaTeX\ source. {\sc Bib}\TeX\ reads the .aux file and matches the citations to the entries in the bibliographic data base file specified in the \LaTeX\ source file by the \verb+\bibliography+ command. {\sc Bib}\TeX\ then writes the bibliography in accordance with the rules in the bibliography .bst style file to a .bbl file which \LaTeX\ merges with the source text.  A good description of the use of {\sc Bib}\TeX\ can be found in \cite{latex, goosens} (see how 2 references are handled?).  The following is an example of how three or more references \cite{latex, asmemanual,  goosens} show up using the \verb+asmems4.bst+ bibliography style file in conjunction with the \verb+asme2e.cls+ class file. Here are some more \cite{art, blt, ibk, icn, ips, mts, mis, pro, pts, trt, upd} which can be used to describe almost any sort of reference.

% Here's where you specify the bibliography style file.
% The full file name for the bibliography style file 
% used for an ASME paper is asmems4.bst.
\bibliographystyle{asmems4}


%%%%%%%%%%%%%%%%%%%%%%%%%%%%%%%%%%%%%%%%%%%%%%%%%%%%%%%%%%%%%%%%%%%%%%
\begin{acknowledgment}
Thanks go to D. E. Knuth and L. Lamport for developing the wonderful word processing software packages \TeX\ and \LaTeX. I also would like to thank Ken Sprott, Kirk van Katwyk, and Matt Campbell for fixing bugs in the ASME style file \verb+asme2e.cls+, and Geoff Shiflett for creating 
ASME bibliography stype file \verb+asmems4.bst+.
\end{acknowledgment}

%%%%%%%%%%%%%%%%%%%%%%%%%%%%%%%%%%%%%%%%%%%%%%%%%%%%%%%%%%%%%%%%%%%%%%
% The bibliography is stored in an external database file
% in the BibTeX format (file_name.bib).  The bibliography is
% created by the following command and it will appear in this
% position in the document. You may, of course, create your
% own bibliography by using thebibliography environment as in
%
% \begin{thebibliography}{12}
% ...
% \bibitem{itemreference} D. E. Knudsen.
% {\em 1966 World Bnus Almanac.}
% {Permafrost Press, Novosibirsk.}
% ...
% \end{thebibliography}

% Here's where you specify the bibliography database file.
% The full file name of the bibliography database for this
% article is asme2e.bib. The name for your database is up
% to you.
\bibliography{asme2e}

%%%%%%%%%%%%%%%%%%%%%%%%%%%%%%%%%%%%%%%%%%%%%%%%%%%%%%%%%%%%%%%%%%%%%%
\appendix       %%% starting appendix
\section*{Appendix A: Head of First Appendix}
Avoid Appendices if possible.

%%%%%%%%%%%%%%%%%%%%%%%%%%%%%%%%%%%%%%%%%%%%%%%%%%%%%%%%%%%%%%%%%%%%%%
\section*{Appendix B: Head of Second Appendix}
\subsection*{Subsection head in appendix}
The equation counter is not reset in an appendix and the numbers will
follow one continual sequence from the beginning of the article to the very end as shown in the following example.
\begin{equation}
a = b + c.
\end{equation}

\end{document}
